%!TEX TS-program = xelatex
%!TEX encoding = UTF-8 Unicode
%!TEX root = 2020-GS-ARTICLE.tex
%-------------------------------------------------------------------------------
%-------------------------------------------------------------------------------
%	PACKAGES AND OTHER DOCUMENT CONFIGURATIONS
%-------------------------------------------------------------------------------
%-------------------------------------------------------------------------------
\documentclass[
	a4paper,
	twocolumn
	]{article}
%--------------------------------------------------------------- GENERAL SETUP -
\usepackage[T1]{fontenc}
\usepackage[italian]{babel}
\usepackage{
  graphicx,
  dblfloatfix
}
\usepackage{graphicx}
\usepackage{epstopdf}
\epstopdfsetup{update}
\usepackage[usenames]{color}
\usepackage{amssymb}
\usepackage{hyperref} % For hyperlinks in the PDF
%---------------------------------------------------------------- STYLE GS2020 -
\input{gs2020.tex}
%-------------------------------------------------------------------- ABSTRACT -
\renewcommand{\maketitlehookd}{%
\begin{abstract}
\noindent\input{includes/abstract.txt}
\end{abstract}
}
%-------------------------------------------------------------------------------
%-------------------------------------------------------------------------------
%	BEGIN DOCUMENT
%-------------------------------------------------------------------------------
%-------------------------------------------------------------------------------
\begin{document}
\maketitle
\thispagestyle{empty}
%-------------------------------------------------------------------------------
%-------------------------------------------------------------------------------
\section*{INTRODUZIONE}
Partendo dal presupposto che la finalità del corso di STALPM 1 fosse quella
di costruire un oggetto multimimediale indirizzato al web e che quindi se ne
potesse usufruire anche in modo "non-locale", si è ragionato fin dalle prime
lezioni sulla tipologia dell'oggetto in questione. Abbiamo quindi effettuato
un veloce brainstorming sicchè la realizzazione dell'oggetto multimediale
potesse sia portarci ad una finalità pratica, cioè quella di potere utilizzare
esso come un nostro strumento di composizione, sia di apprendimento dei concetti
basilari del mondo della programmazione, a noi sconosciuto o quasi fino a quel
momento. Il frutto dei nostri ragionamenti è stato quello di realizzare un synth,
inizialmente con l'idea di basarci su di un modello presistente, poi quella di
realizzarne un modello con una struttura dei vari moduli e percorso del segnale
nuova, rispetto a quelle che erano i nostri pensieri al momento della decisione.
Per la realizzazione dello "Superstereo Synth" si è subito pensato a quale fosse
la strada più efficace da intraprendere per la realizzazione del nostro
sintetizzatore, partendo come detto da una conoscenza quasi nulla rispetto al
mondo della programmazione.  a che tipo di
sintetizzatore realizzare, ponendoci fin da subito la questione
%-------------------------------------------------------------------------------
\subsubsection*{DESCRIZIONE DEL CODICE}
La parte iniziale del codice denominata “GUI” contiene tutti i gruppi grafici dentro i quali abbiamo inserito i vari elementi –grafici appunto- del nostro strumento di sintesi quali knob, fader, pulsanti e meter.

%-----------------------------------------------------
%-------------------------larghezza massima del codice
\begin{lstlisting}
// GUI
main(x) = hgroup("[01] Superstereo Synth",x);
s_g(x) = main(vgroup("[01] SUBTRACTIVE",x));
m_g(x) = main(vgroup("[03] MASTER",x));
p_g(x) = main(vgroup("[05] PHASER",x));
lmeter(x) = main(attach(
	x,an.amp_follower(0.150,x) : ba.linear2db :
	vbargraph("[02] L [unit:dB]", -70,0)));
rmeter(x) = main(attach(
	x,an.amp_follower(0.150,x) : ba.linear2db :
	vbargraph("[04] R [unit:dB]", -70,0)));
meters = lmeter, rmeter;
\end{lstlisting}

Il modulo di sintesi sottrattiva chiamato “SUBTRACTIVE” contiene al suo interno un generatore di rumore bianco e uno di dente di sega con frequenza fissa, swichabili tramite un apposito pulsante. Il tutto va dentro un filtro risonante “Ladder” basato sul modello di Robert Moog con relativi controlli di f cut e Q factor, ed in seguito ad un cotrollo di ampiezza.
Il modulo di sottrattiva è posto nella parte destra del synth.


%-----------------------------------------------------
%-------------------------larghezza massima del codice
\begin{lstlisting}
// SUBTRACTIVE
// Sawfreq Custom
sawfreq = 123;
generator = (no.noise *switch),
	(os.sawtooth(sawfreq)*(1-switch)) :> _;
switch = s_g(checkbox("[01] Saw/Noise")) : si.smoo;
gain = s_g(
	vslider("[04] Gain [style:knob]",-12,-96,+12,0.01))
	: ba.db2linear : si.smoo;
Q = s_g(
	vslider("[03] Q [style:knob]",5,0.7072,25,0.01));
fcut = s_g(
	vslider("[02] Cut [style:knob]",0.65,0,1,0.001)) :
	si.smoo;
subtractive = generator : ve.moogLadder(fcut,Q) :
	*(gain);
\end{lstlisting}


Di seguito troviamo un Phaser Stereo chiamato “PHASERSYNTH” (fig. \ref{phaser}),
costruito tramite una sequenza di N allpass, come dal modello di Curtis Roads su
“The Computer Music Tutorial” \cite{cr96cmt}, di cui il canale R agisce di fase opposta al
canale L. Il segnale verrà reso stereofonico tramite la tecnica dello
“Stereo Shuffling” (fig. \ref{stereoshuffler}). Lo “Superstereophaser”
(così chiamato citando il film “Ritorno al futuro”), avrà 3 tipi di controlli
tramite dei knob:
\begin{compactitem}
\item LFO frequency
\item Feedback
\item Delay
\end{compactitem}


%-----------------------------------------------------
%-------------------------larghezza massima del codice
\begin{lstlisting}
// PHASER
lff = p_g(
	vslider("[02] Lfo [style:knob]",0.35, 0, 16, 0.001))
	: si.smoo;
fbk = p_g(
	vslider("[03] Feedback [style:knob]",
	        -0.689, -0.999, 0.999, 0.001) : si.smoo);
del = p_g(
	nentry("[04] Delay [style:knob]",1, 1, 100, 1));

lfo = os.osc(lff);

phaserLR(N,x,d,g,fb) = x <: l,r
with{
  allpassL(d,g) = (+ <: de.fdelay(
		(ma.SR/2),d),*(-g)) ~ *(g) : mem,_ : +;
  allpassR(d,g) = (+ <: de.fdelay(
		(ma.SR/2),d),*(g)) ~ *(-g) : mem,_ : +;
  apseqL(N,d,g) = seq(i,N,allpassL(d,g));
  apseqR(N,d,g) = seq(i,N,allpassR(d,g));

  l= _<: _, (+:apseqL(N,d,g))~*(fb):> _;
  r= _<: _, (+:apseqR(N,d,g))~*(-fb):> _;
};

// STEREO SHUFFLER
pot = m_g(
	vslider("[02] WIDE [style:knob]",100,0,200,0.1)) :
	/(100) : si.smoo;
somma = + : /(2);
diff = - : /(2);
sdm = somma,diff;
wide = _, * (sqrt(pot));
stereoshuffle= _,_ <: sdm : wide <: sdm;

superstereophaser = phaserLR(4,_,del,lfo,fbk) :
	stereoshuffle;
\end{lstlisting}
%-----------------------------------------------------
%-------------------------larghezza massima del codice

\begin{figure}[h]
\begin{center}
\includegraphics[width=.47\textwidth]{img/phaser}
\caption{\textbf{Phaser}. Realizzato con una serie di N allpass \cite{cr96cmt} .}
\label{phaser}
\end{center}
\end{figure}

\begin{figure}[h]
\begin{center}
\includegraphics[width=.47\textwidth]{img/mid-side-shuffler}
\caption{\textbf{Stereo Shuffler}. blablabla \cite{ab58}.}
\label{stereoshuffler}
\end{center}
\end{figure}

Lo “Superstereophaser” è posto sulla parte destra del synth e potrà essere bypassato tramite un apposito pulsante posto in cima alla sezione stessa.
Successivamente, il nostro segnale entrerà all’interno di un numero N di chopper (Hard Limiter) che limiteranno il segnale in modo netto secondo una soglia prestabilita, seguiti da N filtri passa basso di diverso ordine, in modo da rendere più “smooth” il segnale generato, attenuando le armoniche superiori generate dai chopper e mantenere quelle inferiori.

\begin{lstlisting}
// HARD LIMITER
chopper(a) = min(a) : max(-a);

hardlimiter = chopper(0.7) :
	fi.lowpass(12,15000): chopper(0.9) :
	fi.lowpass6e(20000);
phchop = superstereophaser :
	hardlimiter, hardlimiter;
 \end{lstlisting}

Il tutto entrerà nella sezione chiamata “MASTER CONTROLS” nella quale avremo in ordine:
- Pulsante “MUTE”, che chiude il segnale alla fine della catena (pre-fader)
- Controllo del volume Master tramite un fader verticale con valori in scala logaritmica posto al centro del synth.
-Due Meters, sempre su scala logaritmica, posti al due lati della sezione Master.
La sezione Master si trova nella parte centrale del nostro Synth.

%-----------------------------------------------------
%-------------------------larghezza massima del codice
\begin{lstlisting}
// MASTER CONTROLS
mute = m_g(*(1-(checkbox("[04] Mute")))) : si.smoo;
volume = m_g(vslider("[02] VOLUME ",-6,-70,12,0.1)) :
   ba.db2linear : si.smoo;
bpc = p_g(checkbox("[01] Bypass"));
phaser = ba.bypass1to2(bpc,phchop);
mutes = mute,mute;
master = (*(volume), *(volume));
\end{lstlisting}

Di seguito la struttura generale del programma, sia in forma di codice che
rappresentata attraverso un diagramma a blocchi (fig. \ref{process}):

\begin{lstlisting}
process = subtractive : phaser : master : mutes : meters;
\end{lstlisting}

\begin{figure}[h]
\begin{center}
\includegraphics[width=.47\textwidth]{img/process}
\caption{\textbf{Process}. Diagramma a blocchi della struttura del synth.}
\label{process}
\end{center}
\end{figure}



% \newpage % USE NEWPAGE TO FORCE COLUMNN INTERRUPTION
%-------------------------------------------------------------------------------
%-------------------------------------------------------------------------------
\section*{CONCLUSIONI}

% \begin{quote}
% La musica non e` solo composizione. \\
% Non è artigianato, non è un mestiere. \\
% La musica è pensiero.\cite{nono85}.
% \end{quote}

% \begin{table}[htp]
% \begin{center}
% \begin{tabular}{ll}
% \textbf{Stages} & \textbf{Dur.} \\
% \hline
% \textbf{Omnidirectional Expositions} & 6 mo. \\
% Sound-shape analysis and visualizations & \\
% Sound-shape reproduction & \\
% Sound-shape database design & \\
% \hline
% \textbf{Micro-Rhythm of sound-shape} & 12 mo. \\
% Solo repertoire analysis & \\
% Sound-shape explosion in practising & \\
% From literature to shapes open-data & \\
% \hline
% \textbf{Rhythm of sound-shape interactions} & 12 mo. \\
% Multiple sources multiple shapes & \\
% Relationship and complexity perception & \\
% \hline
% \textbf{Sound-shape in musical composition} & 12 mo. \\
% AI: unleashed writing opportunities & \\
% AI: can you listen the time? & \\
% \hline
% \textbf{Final documentation} & 6 mo. \\
% \end{tabular}
% \label{timesheet}
% \caption{Thinking Tetrahedral Today stages}
% \end{center}
% \end{table}%



\vfill\null

\raggedright
\bibliographystyle{plain}
\bibliography{includes/bibliography.bib}

\end{document}

%%%%%%%%%%%%%%%%%%%%%%%%%%%%%%%%%%%%%%%%%%%%%%%%%%%%%%%%%%%%%%%%%%%%%%%%%%%%%%%%
% 2020 GIUSEPPE SILVI ARTICLE TEMPLATE BASED ON
%%%%%%%%%%%%%%%%%%%%%%%%%%%%%%%%%%%%%%%%%%%%%%%%%%%%%%%%%%%%%%%%%%%%%%%%%%%%%%%%
% Journal Article
% LaTeX Template
% Version 1.4 (15/5/16)
% This template has been downloaded from:
% http://www.LaTeXTemplates.com
% Original author:
% Frits Wenneker (http://www.howtotex.com) with extensive modifications by
% Vel (vel@LaTeXTemplates.com)
% License:
% CC BY-NC-SA 3.0 (http://creativecommons.org/licenses/by-nc-sa/3.0/)
%%%%%%%%%%%%%%%%%%%%%%%%%%%%%%%%%%%%%%%%%%%%%%%%%%%%%%%%%%%%%%%%%%%%%%%%%%%%%%%%
